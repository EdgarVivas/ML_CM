
\documentclass{article}
\usepackage[utf8]{inputenc}
\usepackage{listings}
\usepackage{color}
\usepackage{geometry}
\usepackage{hyperref}
\usepackage{caption}
\usepackage{fancyvrb}

\geometry{a4paper, margin=1in}

% Define colors for listings
\definecolor{codegray}{rgb}{0.5,0.5,0.5}
\definecolor{codepurple}{rgb}{0.58,0,0.82}
\definecolor{backcolour}{rgb}{0.95,0.95,0.92}

\lstdefinestyle{mystyle}{
    backgroundcolor=\color{backcolour},
    commentstyle=\color{codegray},
    keywordstyle=\color{codepurple},
    numberstyle=\tiny\color{codegray},
    stringstyle=\color{codepurple},
    basicstyle=\ttfamily\small,
    breakatwhitespace=false,
    breaklines=true,
    captionpos=b,
    keepspaces=true,
    numbers=left,
    numbersep=5pt,
    showspaces=false,
    showstringspaces=false,
    showtabs=false,
    tabsize=2
}

\lstset{style=mystyle}

\title{Code Listing: 1.1.py}
\author{}
\date{}

\begin{document}

\maketitle

\section*{Source Code}

\begin{lstlisting}[language=Python]
import os
import random
from PIL import Image, ImageDraw
import math

\# Set random seed for reproducibility
random.seed(42)

\# Configuration
IMAGE\_SIZE = (64, 64)  \# Frame size is 64x64
NUM\_IMAGES\_PER\_CLASS = 1000  \# Number of images per class
OUTPUT\_DIR = 'dataset\_bw'  \# Output directory for images

\# Define classes
CLASSES = ['circle\_damage', 'line\_damage', 'star\_damage', 'no\_damage', 'multiple\_damages']

\# Configuration for number of shapes per damage type
SHAPES\_PER\_DAMAGE = \{
    'circle\_damage': (1, 3),  \# 1 to 3 circles per image
    'line\_damage': (1, 3),    \# 1 to 3 lines per image
    'star\_damage': (1, 3),    \# 1 to 3 stars per image
\}

\# Create output directories
def create\_directories(base\_dir, classes):
    if not os.path.exists(base\_dir):
        os.makedirs(base\_dir)
    for cls in classes:
        class\_dir = os.path.join(base\_dir, cls)
        if not os.path.exists(class\_dir):
            os.makedirs(class\_dir)

\# Draw filled circle damage
def draw\_circle(draw):
    radius = random.randint(5, 15)  \# Adjusted for 64x64 frame size
    x = random.randint(radius, IMAGE\_SIZE[0] - radius)
    y = random.randint(radius, IMAGE\_SIZE[1] - radius)
    bounding\_box = [x - radius, y - radius, x + radius, y + radius]
    color = 'white'
    draw.ellipse(bounding\_box, outline=color, fill=color, width=2)

\# Draw line damage
def draw\_line(draw):
    x1 = random.randint(0, IMAGE\_SIZE[0])
    y1 = random.randint(0, IMAGE\_SIZE[1])
    x2 = random.randint(0, IMAGE\_SIZE[0])
    y2 = random.randint(0, IMAGE\_SIZE[1])
    color = 'white'
    width = random.randint(1, 2)  \# Adjusted line width
    draw.line([x1, y1, x2, y2], fill=color, width=width)

\# Draw star damage
def draw\_star(draw):
    num\_points = 5
    outer\_radius = random.randint(8, 20)  \# Adjusted for 64x64 frame size
    x\_center = random.randint(outer\_radius, IMAGE\_SIZE[0] - outer\_radius)
    y\_center = random.randint(outer\_radius, IMAGE\_SIZE[1] - outer\_radius)
    
    rotation\_angle = random.uniform(0, 2 * math.pi)
    points = []
    angle\_offset = rotation\_angle - math.pi / 2
    for i in range(num\_points):
        angle = angle\_offset + (2 * math.pi * i) / num\_points
        x = x\_center + outer\_radius * math.cos(angle)
        y = y\_center + outer\_radius * math.sin(angle)
        points.append((x, y))
    
    connect\_order = [0, 2, 4, 1, 3, 0]
    color = 'white'
    width = 2
    for i in range(num\_points):
        start\_point = points[connect\_order[i]]
        end\_point = points[connect\_order[i + 1]]
        draw.line([start\_point, end\_point], fill=color, width=width)

\# Draw multiple damages
def draw\_multiple(draw):
    damage\_types = ['circle', 'line', 'star']
    num\_damages = random.randint(2, 4)
    for \_ in range(num\_damages):
        damage = random.choice(damage\_types)
        if damage == 'circle':
            draw\_circle(draw)
        elif damage == 'line':
            draw\_line(draw)
        elif damage == 'star':
            draw\_star(draw)

\# Generate no\_damage image
def generate\_no\_damage():
    return Image.new('RGB', IMAGE\_SIZE, color='black')

\# Generate damage image with random number of shapes
def generate\_damage\_image(damage\_type):
    image = Image.new('RGB', IMAGE\_SIZE, color='black')
    draw = ImageDraw.Draw(image)
    
    if damage\_type in SHAPES\_PER\_DAMAGE:
        min\_shapes, max\_shapes = SHAPES\_PER\_DAMAGE[damage\_type]
        num\_shapes = random.randint(min\_shapes, max\_shapes)
        for \_ in range(num\_shapes):
            if damage\_type == 'circle\_damage':
                draw\_circle(draw)
            elif damage\_type == 'line\_damage':
                draw\_line(draw)
            elif damage\_type == 'star\_damage':
                draw\_star(draw)
    elif damage\_type == 'multiple\_damages':
        draw\_multiple(draw)
    
    return image

\# Save images to respective directories
def save\_images():
    create\_directories(OUTPUT\_DIR, CLASSES)
    for cls in CLASSES:
        print(f"Generating images for class: \{cls\}")
        for i in range(NUM\_IMAGES\_PER\_CLASS):
            if cls == 'no\_damage':
                img = generate\_no\_damage()
            else:
                img = generate\_damage\_image(cls)
            img\_filename = f"image\_\{i+1:04d\}.png"
            img\_path = os.path.join(OUTPUT\_DIR, cls, img\_filename)
            img.save(img\_path)
            if (i+1) \% 100 == 0:
                print(f"  Saved \{i+1\} images")
    print("Dataset generation complete.")

if \_\_name\_\_ == "\_\_main\_\_":
    save\_images()

\end{lstlisting}

\end{document}
